\documentclass[10pt,conference]{IEEEtran}
\usepackage[utf8]{inputenc}

% Packages import %%%%%%%%%%%%%%%%%%%%%%%%%%%%%%%%%
\usepackage{graphicx}
\usepackage{verbatim}
\usepackage{hyperref}
\usepackage{booktabs}
\usepackage{multirow}
\usepackage{graphicx}

% Commands definition %%%%%%%%%%%%%%%%%%%%%%%%%%%%%%%%%%%%
% 'image': definição de imagem
\newcommand{\image}[4][\linewidth] {
    \begin{figure}[ht]
    \centering
    \includegraphics[width=#1]{#3}
    \caption{#4}
    \label{#2}
    \end{figure}
}

% 'refimage': referências de imagens
\newcommand{\refimage}[1] {figura \ref{#1}}

% 'refsection': referências de seções
\newcommand{\refsect}[1] {seção \ref{#1}}

% 'reftable': referências de tabelas
\newcommand{\reftable}[1] {tabela \ref{#1}}

\usepackage[english]{babel}

% Quotation mark
\usepackage [autostyle, english = american]{csquotes}
\MakeOuterQuote{"}

% Title
\title{Spatial Temporal Graph Convolutional Networks for Sign Language Recognition}
\date{December 2018}

% Authors
\author{
    \IEEEauthorblockN{Cleison Correia de Amorim}
    \IEEEauthorblockA{
        \textit{Centro de Informática} \\
        \textit{Universidade Federal de Pernambuco} \\
        Recife, PE, Brazil, 50.740-560 \\
        cca5@cin.ufpe.br
    }
    \and
    \IEEEauthorblockN{David Macêdo}
    \IEEEauthorblockA{
        \textit{Centro de Informática} \\
        \textit{Universidade Federal de Pernambuco} \\
        Recife, PE, Brazil, 50.740-560 \\
        dlm@cin.ufpe.br
    }
    \and
    \IEEEauthorblockN{Cleber Zanchettin}
    \IEEEauthorblockA{
        \textit{Centro de Informática} \\
        \textit{Universidade Federal de Pernambuco} \\
        Recife, PE, Brazil, 50.740-560 \\
        cz@cin.ufpe.br
    }
}


\begin{document}
\maketitle

\section{Problema de investigação}
\label{sec:problema-investigacao}

Este projeto propõe a aplicação de técnicas de \textit{deep learning} para a interpretação e transcrição das características fonológicas da língua de sinais para o modelo computacional CORE-SL \cite{antunes-2015}. 

Pode-se observar um grande número de estudos que se propõem a realizar a tradução de diferentes línguas de sinais para a língua falada correspondente. Entretanto, muito desses estudos ficam limitados perante o contexto cotidiano do Surdo por desconsiderarem aspectos relevantes da fonologia da língua de sinais como movimentos, expressões não-manuais, locação e orientação das mãos do interlocutor \cite{quadros-2004}. É comum também que esses estudos restrinjam seu campo de atuação apenas ao âmbito da datilologia\footnote{
Datilologia – também conhecida como alfabeto digital ou alfabeto manual, consiste na soletração manual de palavras pelos Surdos em uma variedade de contextos. É geralmente utilizada para introduzir uma palavra que ainda não possui um sinal equivalente \cite{quadros-2004}\cite{pereira-choi-2011}.
}, que na prática é aplicada em contexto restritos da comunicação do Surdo.

\textcite{antunes-hcisl-2011} apresentam outros fatores que ressaltam as dificuldades de aplicar essas abordagens ao cotidiano do Surdo. De forma sucinta, esses fatores são: 
\begin{enumerate}
\item O uso de equipamentos como luvas, acelerômetros e outros sensores que são de difícil acesso; 
\item A adoção de métodos e tecnologias que não empatizam com a realidade surdo, restringindo sua movimentação ou deixando de considerar aspectos importantes, como as expressões faciais;
\item O uso de métodos que mapeiam sinais diretamente para palavras, e que tornam-se facilmente obsoletos mediante a introdução de novos sinais ou quando confrontados com variações linguísticas como gírias e regionalismos; 
\item A utilização de imagens estáticas para o treinamento de modelos, que desconsideram a dinâmica da língua; 
\end{enumerate}

Diante dessas limitações, \textcite{antunes-hcisl-2011} introduziram a proposta de uma arquitetura capaz de considerar os aspectos fonológicos da língua e de viabilizar a interação homem-máquina por meio dos sinais, a qual foi denominada HCI-SL. A \refimage{fig:hcisl} representa a estrutura proposta pelos pesquisadores, onde uma API interna compreendendo tecnologias de visão computacional e processamento de linguagem natural proveem uma interface comum e padronizada para ferramentas e outras APIs e serviços externos como dicionários, tradutores e aplicações de finalidades diversas.

\image
	[6cm]
    {fig:hcisl}
    {images/hcisl}
    {Arquitetura HCI-SL \cite{antunes-hcisl-2011}}

Para arquitetura acima funcionar, foi necessário primeiro que \textcite{antunes-hcisl-2011} definissem o CORE-SL, que consiste num modelo computacional capaz de descrever completamente os sinais e suas características fonológicas. Além disso, ele tem como papel fundamental estabelecer um padrão de representação dos sinais, passando a ser adotado pelas peças que compõem a arquitetura HCI-SL e por todas as aplicações e serviços desenvolvidos a partir dela (conforme \refimage{fig:coresl-architecture}). 
 
 \image
    [6cm]
    {fig:coresl-architecture}
    {images/hcisl_architecture}
    {Arquitetura do HSI-SL}

De acordo com os pesquisadores, o CORE-SL:

\begin{quote}
[...] agrega flexibilidade e um nível de detalhamento capazes de proporcionar alternativas para um tratamento computacional robusto e para auxiliar às diferentes necessidades de aplicação. Este modelo é fundamental, pois atuará como um dos pilares de sustentação na construção de artefatos tecnológicos que considerem as necessidades deste perfil de usuário e tornem a comunicação usuário-sistema natural para ele. \cite{antunes-2011}
\end{quote}

 A \refimage{fig:coresl-interfaces} ilustra as interações do CORE-SL com uma série de serviços idealizados ou em desenvolvimento por pesquisadores da Universidade Federal do Paraná - UFPR. Nela, o modelo exerce um papel chave para existência e comunicação das demais peças envolvidas.

\image
    {fig:coresl-interfaces}
    {images/coresl_interfaces}
    {CORE-SL como peça chave para a comunicação de outros serviços}

A \refimage{fig:coresl-sinalarvore} mostra um exemplo de descrição do sinal árvore através do modelo CORE-SL:

\image
    {fig:coresl-sinalarvore}
    {images/sinal_arvore}
    {Representação do sinal árvore escrito segundo o CORE-SL}
    

\section{Dados que serão utilizados}
A proposta inicial é utilizar o \textit{American Sign Language Lexicon Video Dataset} (ASLLVD)\footnote{\textit{American Sign Language Lexicon Video Dataset} (ASLLVD) - disponível no endereço \url{http://vlm1.uta.edu/~athitsos/asl_lexicon/}}. Ele é composto por vídeos de sinais realizados por diferentes atores e também por arquivos de metadados que determinam qual o sinal e quais os respectivos frames de início e fim para cada um deles dentro do vídeo. Esse \textit{dataset} é apresentado com mais detalhes em \cite{athitsos-asldataset-2008}.

Apesar do \textit{dataset} acima não se referir especificamente à Libras, sua utilização é válida para este projeto porque o CORE-SL não está atrelado aos sinais, mas às características fonológicas desses sinais, as quais são transversais para línguas de diferentes regiões. 

Uma vez definidos os dados, será necessário produzir os respectivos arquivos de descrição dos sinais em CORE-SL, para que sirvam de entrada no treinamento do modelo a ser elaborado.

Adicionalmente, tem-se como estratégia inicial utilizar a biblioteca OpenPose\footnote{OpenPose - consiste num sistema para detectar simultaneamente pontos chaves do corpo, da mão e do rosto humano (num total de 135) em tempo real para múltiplas pessoas. Está descrito em mais detalhes em \textcite{cao-realtime-2017}, \textcite{simon-hand-2017} e \textcite{wei-cpm-2016} e disponível no endereço \url{https://github.com/CMU-Perceptual-Computing-Lab/openpose}} para extrair os pontos do corpo dos atores presentes no \textit{dataset} acima e produzir uma entrada potencialmente mais rica para o modelo a ser elaborado.

\section{Método a ser utilizado}
Pretende-se elaborar para este projeto um modelo baseado em \textit{Recurrent neural network} - RNN e/ou \textit{Long short-term memory} - LSTM. Entretanto, ainda é necessário expandir o embasamento teórico acerca dessas técnicas e do grau de aplicabilidade ao problema em questão.

Com relação ao modelo CORE-SL, entende-se que a discussão apresentada na \refsect{sec:problema-investigacao} os respectivos trabalhos de citados são suficientes para realização do projeto.

\section{Avaliação dos resultados}
% \section{How will you evaluate your results? Qualitatively, what kind of results do you expect (e.g. plots or figures)? Quantitatively, what kind of analysis will you use to evaluate and/or compare your results (e.g. what performance metrics or statistical tests)?}

\begin{itemize}
\item Quantitativo
\item Qualitativo
\end{itemize}

Dúvida...?


\printbibliography

\end{document}
%%