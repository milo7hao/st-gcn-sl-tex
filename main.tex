\documentclass[10pt,conference]{IEEEtran}
\usepackage[utf8]{inputenc}

% Packages import %%%%%%%%%%%%%%%%%%%%%%%%%%%%%%%%%
\usepackage{graphicx}
\usepackage{verbatim}
\usepackage{hyperref}
\usepackage{booktabs}
\usepackage{multirow}
\usepackage{graphicx}

% Commands definition %%%%%%%%%%%%%%%%%%%%%%%%%%%%%%%%%%%%
% 'image': definição de imagem
\newcommand{\image}[4][\linewidth] {
    \begin{figure}[ht]
    \centering
    \includegraphics[width=#1]{#3}
    \caption{#4}
    \label{#2}
    \end{figure}
}

% 'refimage': referências de imagens
\newcommand{\refimage}[1] {figura \ref{#1}}

% 'refsection': referências de seções
\newcommand{\refsect}[1] {seção \ref{#1}}

% 'reftable': referências de tabelas
\newcommand{\reftable}[1] {tabela \ref{#1}}

\usepackage[english]{babel}

% Quotation mark
\usepackage [autostyle, english = american]{csquotes}
\MakeOuterQuote{"}

% Title
\title{Spatial Temporal Graph Convolutional Networks for Sign Language Recognition}
\date{December 2018}

% Authors
\author{
    \IEEEauthorblockN{Cleison Correia de Amorim}
    \IEEEauthorblockA{
        \textit{Centro de Informática} \\
        \textit{Universidade Federal de Pernambuco} \\
        Recife, PE, Brazil, 50.740-560 \\
        cca5@cin.ufpe.br
    }
    \and
    \IEEEauthorblockN{David Macêdo}
    \IEEEauthorblockA{
        \textit{Centro de Informática} \\
        \textit{Universidade Federal de Pernambuco} \\
        Recife, PE, Brazil, 50.740-560 \\
        dlm@cin.ufpe.br
    }
    \and
    \IEEEauthorblockN{Cleber Zanchettin}
    \IEEEauthorblockA{
        \textit{Centro de Informática} \\
        \textit{Universidade Federal de Pernambuco} \\
        Recife, PE, Brazil, 50.740-560 \\
        cz@cin.ufpe.br
    }
}


\begin{document}
\maketitle

\section{Problema de investigação}
\label{sec:problema-investigacao}

Este projeto propõe a aplicação de técnicas de \textit{deep learning} para a transcrição das características fonológicas da língua de sinais para o modelo computacional CORE-SL proposto por \textcite{antunes-2015}. 

Pode-se observar um número considerável de estudos que se propõem a realizar a tradução de diferentes línguas de sinais para a língua falada correspondente. Entretanto, muito desses estudos ficam limitados perante o contexto cotidiano do Surdo por desconsiderarem aspectos relevantes da fonologia da língua como movimentos, expressões não-manuais, locação e orientação das mãos do interlocutor \cite{quadros-2004}. É comum também que esses estudos restrinjam seu campo de atuação apenas ao âmbito da datilologia\footnote{
Datilologia – também conhecida como alfabeto digital ou alfabeto manual, consiste na soletração manual de palavras pelos Surdos. É geralmente utilizada para introduzir uma palavra que ainda não possui um sinal equivalente \cite{quadros-2004}\cite{pereira-choi-2011}.
}, que na prática é aplicada apenas em contexto restritos da comunicação do Surdo.

\textcite{antunes-hcisl-2011} apresentam alguns outros fatores que justificam a limitação em trazer muitos desses estudos para a realidade dessas pessoas: 
\begin{enumerate}
\item O uso de equipamentos como luvas, acelerômetros e outros sensores que são de difícil acesso; 
\item A adoção de métodos e tecnologias que não empatizam com a realidade surdo, restringindo sua movimentação ou deixando de considerar aspectos importantes, como as expressões faciais;
\item O uso de métodos que mapeiam sinais diretamente para palavras, e que tornam-se facilmente obsoletos mediante a introdução de novos sinais ou quando confrontados com variações linguísticas como gírias e regionalismos; 
\item A utilização de imagens estáticas para o treinamento de modelos, que desconsideram a dinâmica da língua.
\end{enumerate}

Diante dessas limitações, \textcite{antunes-hcisl-2011} introduziram a proposta de uma arquitetura capaz de considerar os aspectos fonológicos da língua e de viabilizar a interação homem-máquina por meio dos sinais, a qual foi denominada HCI-SL. A \refimage{fig:hcisl} apresenta essa arquitetura, onde uma API interna compreendendo tecnologias de visão computacional e processamento de linguagem natural proveem uma interface comum e padronizada para ferramentas e serviços externos como dicionários, tradutores e aplicações de finalidades diversas.

\image
	[6cm]
    {fig:hcisl}
    {images/hcisl}
    {Arquitetura HCI-SL \cite{antunes-hcisl-2011}}

Para a arquitetura acima funcionar, foi necessário primeiro que \textcite{antunes-hcisl-2011} definissem o CORE-SL, que consiste num modelo computacional capaz de descrever completamente os sinais e suas características fonológicas. Além disso, ele possui como papel fundamental o estabelecimento de um padrão para representação dos sinais, a ser adotado pelas peças que compõem a arquitetura HCI-SL e pelas aplicações e serviços desenvolvidos a partir dela.

De acordo com os pesquisadores, o CORE-SL:

\begin{quote}
[...] agrega flexibilidade e um nível de detalhamento capazes de proporcionar alternativas para um tratamento computacional robusto e para auxiliar às diferentes necessidades de aplicação. Este modelo atuará como um dos pilares de sustentação na construção de artefatos tecnológicos que considerem as necessidades deste perfil de usuário e tornem a comunicação usuário-sistema natural para ele. \cite{antunes-2011}
\end{quote}

 A \refimage{fig:coresl-interfaces} ilustra as interações do CORE-SL com um conjunto de serviços idealizados ou em desenvolvimento por pesquisadores da Universidade Federal do Paraná - UFPR. Nela, o modelo exerce um papel central para a comunicação das peças envolvidas.

\image
    {fig:coresl-interfaces}
    {images/coresl_interfaces}
    {CORE-SL como peça chave para a comunicação de outros serviços}

A \refimage{fig:coresl-sinalarvore} mostra um exemplo de descrição do sinal árvore através do modelo CORE-SL.

\image
    {fig:coresl-sinalarvore}
    {images/sinal_arvore}
    {Representação do sinal árvore escrito segundo o CORE-SL}
    

\section{Dados que serão utilizados}
\label{sec:dados-utilizados}
A proposta inicial é utilizar o \textit{American Sign Language Lexicon Video Dataset} (ASLLVD)\footnote{\textit{American Sign Language Lexicon Video Dataset} (ASLLVD) - disponível no endereço \url{http://vlm1.uta.edu/~athitsos/asl_lexicon/}}. Ele é composto por vídeos de sinais realizados por diferentes atores e também por arquivos de metadados que determinam qual o sinal e quais os respectivos frames de início e fim para cada um deles dentro do vídeo. Esse \textit{dataset} é apresentado com mais detalhes em \cite{athitsos-asldataset-2008}.

Apesar do \textit{dataset} acima não se referir especificamente à Libras, sua utilização é válida para este projeto porque o CORE-SL não está atrelado aos sinais propriamente ditos, mas sim às suas características fonológicas, as quais são universais para línguas de sinais de diferentes regiões. 

Uma vez definidos os dados, o segundo passo consiste em produzir as respectivas \textit{labels} (ou arquivos de descrição dos sinais em CORE-SL) para cada um dos vídeos do \textit{dataset}, para que sirvam de referência no treinamento do modelo. O pouco tempo disponível para elaboração deste projeto, entretanto, atua como limitador na quantidade de dados com \textit{labels} passíveis de utilização pelo modelo neste primeiro momento. 

Adicionalmente, tem-se como estratégia utilizar a biblioteca OpenPose para extrair os pontos do corpo dos atores presentes no \textit{dataset} acima e produzir uma entrada potencialmente mais rica para o modelo. A OpenPose é capaz de detectar os atores e fornecer cerca de 135 pontos referentes a partes de seus corpos, como mãos, braços e rosto, conforme descrito em \textcite{cao-realtime-2017}, \textcite{simon-hand-2017}, \textcite{wei-cpm-2016} e no endereço \url{https://github.com/CMU-Perceptual-Computing-Lab/openpose}. Apesar disso, o acesso à quantidade considerável de recursos computacionais requeridos por essa biblioteca durante a execução do projeto (como por exemplo, GPUs com mais de 4GB de memória), pode ser um fator limitador em potencial para sua aplicação neste momento.

\section{Método a ser utilizado}
Pretende-se elaborar para este projeto um modelo baseado em \textit{Recurrent neural network} - RNN e técnicas derivadas (como o \textit{Long short-term memory} - LSTM). Entretanto, ainda é necessário expandir o embasamento teórico acerca dessas técnicas e do grau de aplicabilidade ao problema em questão. 

Com relação ao modelo CORE-SL, entende-se que as referências utilizadas na discussão apresentada na \refsect{sec:problema-investigacao} são suficientes para a condução do projeto.

\section{Avaliação dos resultados}
Devido à restrição no número de amostras com \textit{labels} em CORE-SL que serão geradas (conforme \refsect{sec:dados-utilizados}), a estratégia inicial para validação do modelo consiste em utilizar de técnicas de validação cruzada como o K-Fold, pela sua capacidade de maximizar o uso dos dados disponíveis através de diferentes combinações entre eles.

Com relação às métricas a serem adotadas, ainda há necessidade de pesquisa quanto àquelas mais apropriadas ao tipo de modelo RNN e, principalmente, ao tipo de problema proposto acima. Apesar disso, sabe-se de antemão que essas métricas devem ser capazes de mensurar o grau de similaridade (ou proporção do acerto/erro) das saídas produzidas pelo modelo, dado que um \textit{label} em CORE-SL consiste em um XML contendo diversas características fonológicas que permitem um julgamento do tipo "quão correta está a descrição do sinal produzida pelo modelo", "quantas das características estão corretas?" ou "o XML obtido está 90\% semelhante com relação ao XML de referência". Métricas que limitem a avaliação de uma forma binária como "certo" ou "errado" não se mostram adequadas para este problema.


\printbibliography

\end{document}
%%