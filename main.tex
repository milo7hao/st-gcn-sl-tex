\documentclass[9pt,conference]{IEEEtran}
\usepackage[utf8]{inputenc}
\usepackage[brazil]{babel}

% Diversos
\usepackage{csquotes}
\usepackage{graphicx}
\usepackage{verbatim}
\usepackage{hyperref}
\usepackage{smartdiagram}

% Título
\title{Utilizando Redes Convolucionais de Grafos Espaço-Temporais para o Reconhecimento da Línguas de Sinais}
%\author{Cleison Correia de Amorim}
\date{Outubro 2018}

\author{
    \IEEEauthorblockN{Cleison Correia de Amorim}
    \IEEEauthorblockA{Centro de Informática\\
    Universidade Federal de Pernambuco\\
    Email: cca5@cin.ufpe.br}
}

% Comandos
% 'image': definição de imagem
\newcommand{\image}[4][\linewidth] {
    \begin{figure}[ht]
    \centering
    \includegraphics[width=#1]{#3}
    \caption{#4}
    \label{#2}
    \end{figure}
}

% 'refimage': referências de imagens
\newcommand{\refimage}[1] {figura \ref{#1}}

% 'refsection': referências de seções
\newcommand{\refsect}[1] {seção "\nameref{#1}"}

\begin{document}
\maketitle

\section{What is the problem that you will be investigating? Why is it interesting?}

Este projeto propõe a aplicação de técnicas de \textit{deep learning} para a transcrição dos movimentos da Libras para dentro do modelo computacional CORE-SL, que é capaz de descrever aspectos de sua fonologia e foi proposto por \cite{antunes-2015}. 

Pode-se observar um grande número de estudos que se propõem a realizar a tradução de diferentes línguas de sinais para a língua falada correspondente. Entretanto, muito desses estudos ficam limitados perante o contexto cotidiano do Surdo por desconsiderarem aspectos relevantes da fonologia da língua de sinais como movimentos, expressões não-manuais, locação e orientação das mãos do interlocutor \cite{quadros-2004}. É comum também que esses estudos restrinjam seu campo de atuação apenas ao âmbito da datilologia\footnote{
Datilologia – também conhecida como alfabeto digital ou alfabeto manual, consiste na soletração manual de palavras pelos Surdos em uma variedade de contextos. É geralmente utilizada para introduzir uma palavra que ainda não possui um sinal equivalente \cite{quadros-2004}\cite{pereira-choi-2011}.
}, que na prática é aplicada em contexto restritos da comunicação do Surdo.

\textcite{antunes-hcisl-2011} apresentam outros fatores que ressaltam as dificuldades de aplicar essas abordagens ao cotidiano do Surdo. De forma sucinta, esses fatores são: 
\begin{enumerate}
\item O uso de equipamentos como luvas, acelerômetros e outros sensores que são de difícil acesso; 
\item A adoção de métodos e tecnologias que não empatizam com a realidade surdo, restringindo sua movimentação ou deixando de considerar aspectos importantes, como as expressões faciais;
\item O uso de métodos que mapeiam sinais diretamente para palavras, e que tornam-se facilmente obsoletos mediante a introdução de novos sinais ou quando confrontados com variações linguísticas como gírias e regionalismos; 
\item A utilização de imagens estáticas para o treinamento de modelos, que desconsideram a dinâmica da língua; 
\end{enumerate}

Diante dessas limitações, \textcite{antunes-hcisl-2011} introduziram a proposta de uma arquitetura capaz de considerar os aspectos fonológicos da língua e de viabilizar a interação homem-máquina por meio dos sinais, a qual foi denominada HCI-SL. A figura \ref{fig:hcisl} representa a estrutura proposta pelos pesquisadores, onde uma API interna compreendendo tecnologias de visão computacional e processamento de linguagem natural proveem uma interface comum e padronizada para ferramentas e outras APIs e serviços externos como dicionários, tradutores e aplicações de finalidades diversas.

\image
	[6cm]
    {fig:hcisl}
    {images/hcisl}
    {Arquitetura HCI-SL \cite{antunes-hcisl-2011}}

Para arquitetura acima funcionar, foi necessário primeiro que \textcite{antunes-hcisl-2011} definissem um modelo computacional capaz de descrever completamente os sinais e suas características fonológicas. Além disso, esse modelo teria como papel fundamental estabelecer um padrão de representação da língua de sinais, passando a ser adotado pelas peças que compõem a arquitetura HCI-SL, bem como por todas aplicações e serviços desenvolvidos a partir dela. Os pesquisadores definem o CORE-SL como sendo:

\begin{quote}
Neste sentido, o trabalho apresenta o desenvolvimento de um modelo para a descrição computacional dos aspectos fonológicos dos sinais, que agrega flexibilidade e um nível de detalhamento capazes de proporcionar alternativas para um tratamento computacional robusto e para auxiliar às diferentes necessidades de aplicação. Este modelo é fundamental, pois atuará como um dos pilares de sustentação na construção de artefatos tecnológicos que considerem as necessidades deste perfil de usuário e tornem a comunicação usuário-sistema natural para ele. \cite{antunes-2011}
\end{quote}

\image
    {fig:ligacoes_coresl}
    {images/ligacoes_coresl}
    {Ligações do CORE-SL}




\image
    [8cm]
    {fig:hcisl_architecture}
    {images/hcisl_architecture}
    {Arquitetura do HSI-SL}


\section{What data will you use? If you are collecting new datasets, how do you plan to collect
them?}

\section{What method or algorithm are you proposing? If there are existing implementations, will you use them and how? How do you plan to improve or modify such implementations?}

\section{What reading will you examine to provide context and background?}

\section{How will you evaluate your results? Qualitatively, what kind of results do you expect (e.g. plots or figures)? Quantitatively, what kind of analysis will you use to evaluate and/or compare your results (e.g. what performance metrics or statistical tests)?}

\printbibliography

\end{document}
%%